\documentclass{resume}
\usepackage{zh_CN-Adobefonts_external} 
\usepackage{linespacing_fix}
\usepackage{cite}
\begin{document}
\pagenumbering{gobble}


%***"%"后面的所有内容是注释而非代码,不会输出到最后的PDF中
%***使用本模板,只需要参照输出的PDF,在本文档的相应位置做简单替换即可
%***修改之后,输出更新后的PDF,只需要点击Overleaf中的“Recompile”按钮即可


%**********************************姓名********************************************
\name{任世伟}


%**********************************联系信息****************************************
%第一个括号里写手机号,第二个写邮箱
% \contactInfo{1234567890}{rsw1835@email.com}
\otherInfo{手机:1234567890}{邮箱:rsw1835@email.com}{}{}


%**********************************其他信息****************************************
%在大括号内填写其他信息,最多填写4个,但是如果选择不填信息,
%那么大括号必须空着不写,而不能删除大括号。
%\otherInfo后面的四个大括号里的所有信息都会在一行输出
%如果想要写两行,那就用两次这个指令(\otherInfo{}{}{}{})即可
\otherInfo{性别:男}{民族:汉}{出生年月:1993.11}{}
\otherInfo{籍贯:河北省石家庄市}{}{}{}


%*********************************照片**********************************************
%照片需要放到images文件夹下,名字必须是you.jpg,如果不需要照片可以不添加此行命令
%0.15的意思是,照片的宽度是页面宽度的0.15倍,调整大小,避免遮挡文字
\yourphoto{0.15}


%**********************************正文**********************************************
%***大标题,下面有横线做分割
%***一般的标题有:教育背景,实习(项目)经历,工作经历,自我评价,求职意向,等等
\section{教育经历}


%***********一行子标题**************
%***第一个大括号里的内容向左对齐,第二个大括号里的内容向右对齐
%***\textbf{}括号里的字是粗体,\textit{}括号里的字是斜体
\datedsubsection{\textbf{上海交通大学} \qquad 计算机技术 \qquad\qquad \textit{硕士}}{2019.09 - 至今}


%***********列举*********************
%***可添加多个\item,得到多个列举项,类似的也可以用\textbf{}、\textit{}做强调
\begin{itemize}[parsep=1ex]
  \item \textbf{主要课程}:算法设计与分析、高级数据库技术、区块链技术、计算机网络、人工智能、数字图像处理、计算机安全学、数据分析
\end{itemize}

\datedsubsection{\textbf{华南理工大学} \qquad 热能与动力工程 \qquad \textit{本科}}{2012.09 - 2016.06}
\begin{itemize}[parsep=1ex]
  \item \textbf{主要课程}:高等数学、线性代数与解析几何、概率论与数理统计、大学计算机基础、C++程序设计
\end{itemize}

\section{技能专长}

\begin{itemize}[parsep=1ex]
  \item \textbf{语言}:Python、R、C/C++、Java、SQL
  \item \textbf{项目/框架}:NumPy、pandas、PyTorch、TensorFlow
\end{itemize}

\section{项目经历}

\datedsubsection{\textbf{在线故障预测与智能调度}}{2020.01 - 至今}
\begin{itemize}[parsep=1ex]
  \item \textbf{项目描述}:在大规模集群中进行实时异常检测,根据异常检测结果和迁移代价分析,做出负载均衡调度。
  \item \textbf{职责/工作}:
    \begin{itemize}[parsep=0.5ex]
      \item 构建基于深度学习的增量异常检测模型;
      \item 构建基于深度学习的指标预测模型;
      \item 结合异常检测模型和指标预测模型,提高异常检测的准确性和实时性。
    \end{itemize}
\end{itemize}

\datedsubsection{\textbf{Style transfer based on Deep Learning}}{2019.10 - 2019.12}
\begin{itemize}[parsep=1ex]
  \item \textbf{项目描述}:利用深度学习算法学习著名画作的风格,然后再把这种风格应用到其他图片上。
  \item \textbf{职责/工作}:
    \begin{itemize}[parsep=0.5ex]
      \item 使用TensorFlow框架定义图像生成网络、内容损失和风格损失;
      \item 结合预训练好的损失网络VGGNet,训练图像生成网络。
    \end{itemize}
\end{itemize}

\datedsubsection{\textbf{软件开发投入工作量分析}}{2019.10 - 2019.11}
\begin{itemize}[parsep=1ex]
  \item \textbf{项目描述}:对软件开发项目数据进行统计分析,得到项目开发时间关于其它影响因素的模型。
  \item \textbf{职责/工作}:
    \begin{itemize}[parsep=0.5ex]
      \item 对软件开发项目数据进行处理;
      \item 分析类别变量的不同水平对项目开发时间的影响(方差分析)。
    \end{itemize}
\end{itemize}

\datedsubsection{\textbf{可编程传输路径}}{2019.09 - 2019.10}
\begin{itemize}[parsep=1ex]
  \item \textbf{项目描述}:在移动性管理(Mobile Management)中对转发路径进行编程,在配置信息更少、信令开销更小的情况下灵活地切换路径,为各种网络服务的应用提供保证。
  \item \textbf{职责/工作}:
    \begin{itemize}[parsep=0.5ex]
      \item 用户面信令(GTP-U)扩展、时延抖动分析。
    \end{itemize}
\end{itemize}

\section{个人特质}

\begin{itemize}[parsep=1ex]
  \item{热爱编程,熟练掌握Python、R,了解C/C++、Java、SQL;}
  \item{有很强的进取心,对新事物充满好奇,熟练掌握PyTorch、TensorFlow等深度学习框架;}
  \item{熟悉基本的数据结构和算法,希望能在互联网行业有所成就。}
\end{itemize}

% \section{简历写作注意事项}

% 写作时不要泛泛而谈太笼统,要应用STAR原则,即Situation(情景)、Task(任务)、Action(行动)和Result(结果)四个英文单词的首字母组合。

% \begin{itemize}[parsep=1ex]
%   \item S指的是situation,事情是在什么情况下发生
%   \item T指的是task,你是如何明确你的目标的
%   \item A指的是action,针对这样的情况分析,你采用了什么行动方式
%   \item R指的是result,结果怎样,在这样的情况下你学习到了什么
% \end{itemize}

\end{document}
